% tex created by table_ctrfSpills.py 

    \begin{table}[!h]
    \caption{Spillovers}
    \label{table:ctrf-spillovers}
    \begin{center}
    \begin{tabular}{ccccccc}
    \multirow{2}{2cm}{Campaign size (\%)}  
    & \multirow{2}{2cm}{Predicted smoking} 
    & \multicolumn{4}{c}{Predicted effect}
    & \multirow{2}{*}{Multiplier} \\
    & & Model
    & Exog net
    & No network
    & Prop    
    \\
     \hline \hline
    - & 42.1 & - & - & - & - & \\ 
   3& 39.6&  2.6&  2.0&  2.0&  1.2&  2.0 \\ 
   5& 38.2&  3.9&  3.1&  3.1&  1.9&  1.9 \\ 
  10& 34.6&  7.5&  5.9&  6.1&  3.5&  1.8 \\ 
  20& 28.7& 13.4& 10.7& 11.0&  5.7&  1.6 \\ 
  30& 23.5& 18.6& 14.9& 15.4&  7.1&  1.5 \\ 
  50& 15.1& 27.0& 21.9& 22.4&  7.5&  1.3 \\ 

    \hline
    \end{tabular}
    \end{center}
    \fignotetitle{Note:} The first and the second columns list the alternative attendance rates and the simulated smoking prevalences respectively.
    Columns three to six display the simulated decrease in overall smoking for different estimation scenarios:
        the full model, the model with exogenous (fixed) social network, the model with no social network data, and 
        the policy effect if it were (only) proportional to the intervention, i.e. a baseline without peer effects. 
        The last column computes the ratio between the percentage change in the number of smokers and the attendance rate.
        %Note that that attendance is random with respect to the smoking status of the students. 
        %If the campaign is able to target only students who are currently smokers, the spillover effects will be even larger.
    \end{table}
    